{\bf\Large Open-Source \textcolor{my_blue}{Contributions}}\\ \vspace*{-6mm}

\begin{itemize}
    \small
    \item \textbf{Parallel Tempering Monte Carlo (PTMC):}
          This project introduces an advanced Fortran (2003) code that uses the Parallel Tempering Monte Carlo (PTMC) method for an accurate and efficient analysis of phase transitions in atomic and molecular clusters. Our application includes studying melting transitions of noble gases and water clusters, even in ultra-high magnetic fields. The code is parallelized using OpenMP and it is optimized for HPC architectures. The code is open-source and available on \href{https://github.com/e-florez/PTMC}{github.com/e-florez/PTMC}

    \item \textbf{Atomic and Molecular Cluster Energy Surface Sampler (AMCESS):}
          An open-source Python package that automates the exploration of the Potential Energy Surface (PES) for atomic and molecular clusters. AMCESS uses advanced optimization techniques to generate candidate structures for critical points on the PES, enhancing research accuracy and efficiency. Its user-friendly design and integration with popular quantum chemistry packages make it a valuable tool for researchers in molecular physics, chemistry, and materials science. AMCESS code is available on \href{https://gitlab.com/ADanianZE/amcess}{gitlab.com/ADanianZE/amcess}

    \item \textbf{Cluster Compare (pyCC):}
          Python project designed to analyze and compare atomic and molecular clusters. The main script is created to orchestrate a rigorous and insightful comparison of cluster data, facilitating a deeper understanding of their intrinsic properties and interactions. pyCC uses a range of data analysis and machine learning modules to derive key cluster features from atomic coordinates and bond patterns, including radial and angular distributions, and various similarity descriptors. By harnessing libraries like numpy, scipy, pandas, and scikit-learn, it provides a specialized, versatile framework for a comparative study of atomic and molecular clusters. The code is open-source and available on \href{https://github.com/e-florez/pyCC}{github.com/e-florez/pyCC}

\end{itemize}